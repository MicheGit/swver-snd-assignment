\section{The Language}

\begin{frame}{The Language}

    The language is a variation of the While language seen in class. It differs on:
    
    \begin{itemize}
        \item it admits some syntactic sugar (it's not minimal);
        \item its semantic functions model divergence and state changes in both arithmetic and boolean expressions.
    \end{itemize}
    
\end{frame}

\subsection{Arithmetic Expressions}

\begin{frame}{Arithmetic Expressions - Syntax}
    \begin{flalign*}
        AExp ::= & n \pipe x \pipe \texttt{-}e \pipe (e)  \\
        \pipe & e_1 + e_2 \pipe e_1 - e_2 \pipe e_1 * e_2 \pipe e_1 / e_2  \\
        \pipe & x\texttt{++} \pipe \texttt{++}x \pipe x\texttt{--} \pipe \texttt{--}x 
    \end{flalign*}

    The syntax allows arithmetic expression that change the state, such as \texttt{x++} and \texttt{x--}.

    The operator $\mathtt{(\cdot/\cdot)}: \mathbb{N} \to \mathbb{N} \to \mathbb{N}$ returns the quotient of the two arguments. It's undefined when the second argument is $0$.
\end{frame}
    
\begin{frame}{Arithmetic Expressions - Semantics (1)}

    \begin{exampleblock}{$\mathcal{A} : AExp \to State \hookrightarrow \mathbb{Z} \times State $}
        \begin{flalign*}
            \semantics{A}{n} \varphi = & (n_\mathbb{Z}, \varphi) \\
            \semantics{A}{x} \varphi = & (\varphi(x), \varphi) \\
            \semantics{A}{(e)} \varphi = & \semantics{A}{e} \varphi \\
            \semantics{A}{-e} \varphi = & 
            \begin{cases}
                (-a, \varphi') & \semantics{A}{e} \varphi = (a, \varphi') \\
                \uparrow & (\semantics{A}{e} \varphi)\uparrow 
            \end{cases} \\
        \end{flalign*}
    \end{exampleblock}
\end{frame}

\begin{frame}{Arithmetic Expressions - Semantics (2)}
    \begin{exampleblock}{$\mathcal{A} : AExp \to State \hookrightarrow \mathbb{Z} \times State $}
        \begin{flalign*}
            \semantics{A}{e_1 / e_2} \varphi = & \begin{cases}
                (a_1 \div a_2, \varphi'') & \semantics{A}{e_1}\varphi = (a_1, \varphi') \\
                & \land\,\,\semantics{A}{e_2}\varphi' = (a_2, \varphi'') \\
                & \land\,\,a_2 \neq 0\\
                \uparrow & \text{otherwise}
            \end{cases}\\
            \semantics{A}{e_1\,\mathbf{op}\,e_2} \varphi = & \begin{cases}
                (a_1\,\,op\,\,a_2, \varphi'') & \semantics{A}{e_1}\varphi = (a_1, \varphi') \\
                & \land\,\,\semantics{A}{e_2}\varphi' = (a_2, \varphi'') \\
            \uparrow & \text{otherwise}
            \end{cases} \\
        \end{flalign*}
    \end{exampleblock}
\end{frame}


\begin{frame}{Arithmetic Expressions - Semantics (3)}
    \begin{exampleblock}{$\mathcal{A} : AExp \to State \hookrightarrow \mathbb{Z} \times State $}
        \begin{flalign*}
            \semantics{A}{x\texttt{++}} \varphi = & (\varphi(x), \varphi[x \mapsto x+1]) \\
            \semantics{A}{\texttt{++}x} \varphi = & let\,\,\varphi' = \varphi[x \mapsto x+1] \\
            & in\,\,(\varphi'(x), \varphi') \\
            \semantics{A}{x\texttt{--}} \varphi = & (\varphi(x), \varphi[x \mapsto x-1]) \\
            \semantics{A}{\texttt{--}x} \varphi = & let\,\,\varphi' = \varphi[x \mapsto x-1] \\
            & in\,\,(\varphi'(x), \varphi') \\
        \end{flalign*}
    \end{exampleblock}
\end{frame}

\subsection{Boolean Expressions}

\begin{frame}{Boolean Expressions - Syntax}
    
    \begin{flalign*}
        BExp ::= & \texttt{true} \pipe \texttt{false} \pipe (b) \pipe b_1\,\,\texttt{and}\,\,b_2 \pipe b_1\,\,\texttt{or}\,\,b_2 \\ \pipe & e_1 \texttt{ = } e_2 \pipe e_1 \texttt{ != } e_2 \pipe e_1 \texttt{ < } e_2 \pipe e_1 \texttt{ >= } e_2  \\
        \pipe & e_1 \texttt{ > } e_2 \pipe e_1 \texttt{ <= } e_2
    \end{flalign*}

    The operator $(\neg \cdot): \mathbb{T} \to \mathbb{T}$ is not in the minimal definition: it is defined as syntactic sugar later on.
\end{frame}

\begin{frame}{Boolean Expressions - Semantics (1)}
    Operators between booleans short-circuit evaluation:
    \begin{exampleblock}{$\mathcal{B} : BExp \to State \hookrightarrow \mathbb{T} \times State$}
        \begin{flalign*}
            \semantics{B}{\texttt{true}} \varphi = & (\mathbf{tt}, \varphi) \\
            \semantics{B}{\texttt{false}} \varphi = & (\mathbf{ff}, \varphi) \\
            \semantics{B}{(b)} \varphi = & \semantics{B}{b} \varphi\\
            \semantics{B}{b_1\,\,\texttt{and}\,\,b_2} \varphi = & \begin{cases}
                (\mathbf{ff}, \varphi') & \semantics{B}{b_1} \varphi = (\mathbf{ff}, \varphi') \\
                \semantics{B}{b_2} \varphi' & \semantics{B}{b_1} \varphi = (\mathbf{tt}, \varphi') \\
                \uparrow & \text{otherwise}
            \end{cases}\\
            \semantics{B}{b_1\,\,\texttt{or}\,\,b_2} \varphi = & \begin{cases}
                (\mathbf{tt}, \varphi') & \semantics{B}{b_1} \varphi = (\mathbf{tt}, \varphi') \\
                \semantics{B}{b_2} \varphi' & \semantics{B}{b_1} \varphi = (\mathbf{ff}, \varphi') \\
                \uparrow & \text{otherwise}
            \end{cases}\\
        \end{flalign*}
        
    \end{exampleblock}
\end{frame}

\begin{frame}{Boolean Expressions - Semantics (2)}
    Comparison operators propagate the state transition(s):

    \begin{exampleblock}{$\mathcal{B} : BExp \to State \hookrightarrow \mathbb{T} \times State$}
        \begin{flalign*}
            \semantics{B}{e_1\texttt{ = }e_2} \varphi = &\begin{cases}
                (a_1 = a_2, \varphi'') & \semantics{A}{e_1} \varphi = (a_1, \varphi') \\
                & \land \,\, \semantics{A}{e_2} \varphi' = (a_2, \varphi'') \\
                \uparrow & \text{otherwise}
            \end{cases}\\
            \semantics{B}{e_1\texttt{ != }e_2} \varphi = &\begin{cases}
                (a_1 \neq a_2, \varphi'') & \semantics{A}{e_1} \varphi = (a_1, \varphi') \\
                & \land \,\, \semantics{A}{e_2} \varphi' = (a_2, \varphi'') \\
                \uparrow & \text{otherwise}
            \end{cases}\\
        \end{flalign*}
    \end{exampleblock}
\end{frame}


\begin{frame}{Boolean Expressions - Semantics (3)}
    \begin{exampleblock}{$\mathcal{B} : BExp \to State \hookrightarrow \mathbb{T} \times State$}
        \begin{flalign*}
            \semantics{B}{e_1\texttt{ < }e_2} \varphi = & \begin{cases}
                (a_1 < a_2, \varphi'') & \semantics{A}{e_1} \varphi = (a_1, \varphi') \\
                & \land \,\, \semantics{A}{e_2} \varphi' = (a_2, \varphi'') \\
                \uparrow & \text{otherwise}
            \end{cases}\\
            \semantics{B}{e_1\texttt{ >= }e_2} \varphi = & \begin{cases}
                (a_1 \geq a_2, \varphi'') & \semantics{A}{e_1} \varphi = (a_1, \varphi') \\
                & \land \,\, \semantics{A}{e_2} \varphi' = (a_2, \varphi'') \\
                \uparrow & \text{otherwise}
            \end{cases}\\
        \end{flalign*}
    \end{exampleblock}
\end{frame}

\begin{frame}{Boolean Expressions - Semantics (4)}
    \begin{exampleblock}{$\mathcal{B} : BExp \to State \hookrightarrow \mathbb{T} \times State$}
        \begin{flalign*}
            \semantics{B}{e_1\texttt{ > }e_2} \varphi = &\begin{cases}
                (a_1 > a_2, \varphi'') & \semantics{A}{e_1} \varphi = (a_1, \varphi') \\
                & \land \,\, \semantics{A}{e_2} \varphi' = (a_2, \varphi'') \\
                \uparrow & \text{otherwise}
            \end{cases}\\\\
            \semantics{B}{e_1\texttt{ <= }e_2} \varphi = & \begin{cases}
                (a_1 \leq a_2, \varphi'') & \semantics{A}{e_1} \varphi = (a_1, \varphi') \\
                & \land \,\, \semantics{A}{e_2} \varphi' = (a_2, \varphi'') \\
                \uparrow & \text{otherwise}
            \end{cases}\\
        \end{flalign*}
    \end{exampleblock}
\end{frame}

\begin{frame}{Boolean Expressions - Syntactic Sugar}
    \begin{alertblock}{Rule}
        Since boolean expressions induce state transitions, the evaluation order and quantity must be preserved in the desugared code.
    \end{alertblock}

    This is the reason why we couldn't model the operators $\mathbb{(\cdot>\cdot), (\cdot\leq\cdot): \mathbb{N} \to \mathbb{N} \to \mathbb{T}}$ as syntactic sugar. There is no way to encode those operators only with $(\cdot<\cdot),(\cdot>=\cdot),(\cdot=\cdot)$ and $(\cdot!=\cdot)$ respecting this rule.
    
\end{frame}

\begin{frame}{Boolean Expressions - Negation}

    \small\begin{flalign*}
        \texttt{not}\,\,\texttt{true} &\defas \texttt{false} \\
        \texttt{not}\,\,\texttt{false} &\defas \texttt{true} \\
        \texttt{not}\,\,(b_1\,\,\texttt{and}\,\,b_2) &\defas \texttt{not}\,\,b_1\,\,\texttt{or}\,\,\texttt{not}\,\,b_2 \\
        \texttt{not}\,\,(b_1\,\,\texttt{or}\,\,b_2) &\defas \texttt{not}\,\,b_1\,\,\texttt{and}\,\,\texttt{not}\,\,b_2 \\
        \texttt{not}\,\,e_1\texttt{ = }e_2 &\defas e_1\texttt{ != }e_2 \\
        \texttt{not}\,\,e_1\texttt{ != }e_2 &\defas e_1\texttt{ = }e_2 \\
        \texttt{not}\,\,e_1\texttt{ < }e_2 &\defas e_1\texttt{ >= }e_2 \\
        \texttt{not}\,\,e_1\texttt{ >= }e_2 &\defas e_1\texttt{ < }e_2 \\
        \texttt{not}\,\,e_1\texttt{ > }e_2 &\defas e_1\texttt{ <= }e_2 \\
        \texttt{not}\,\,e_1\texttt{ <= }e_2 &\defas e_1\texttt{ > }e_2 \\
    \end{flalign*}
\end{frame}


\begin{frame}{Statements (1)}
    \begin{flalign*}
        \mathbf{While} ::= &\,\,x\texttt{ := }e \pipe \texttt{ skip } \pipe \{S\} \pipe S_1\texttt{ ; }S_2 \\ 
        \pipe & \texttt{ if }b\texttt{ then }S_1\texttt{ else }S_2 \pipe \texttt{ while }b\texttt{ do }S 
    \end{flalign*}

    \begin{exampleblock}{$\mathcal{S}_{ds} : While \to State \hookrightarrow State$}
        \begin{flalign*}
            \semantics[ds]{S}{x\texttt{ := } e} \varphi = & \begin{cases}
                \varphi'[x \mapsto a] & \semantics{A}{e} \varphi = (a, \varphi') \\
                \uparrow & \text{otherwise}
            \end{cases} \\
            \semantics[ds]{S}{\texttt{skip}} \varphi = & \varphi \\
            \semantics[ds]{S}{\{S\}} \varphi = & \semantics[ds]{S}{S} \varphi\\
        \end{flalign*}
        
    \end{exampleblock}
\end{frame}

\begin{frame}{Statements (2)}
    \begin{exampleblock}{$\mathcal{S}_{ds} : While \to State \hookrightarrow State$}
        \begin{flalign*}
            \semantics[ds]{S}{S_1\,\texttt{;}\,S_2} \varphi = & (\semantics[ds]{S}{S_2} \circ \semantics[ds]{S}{S_1}) \varphi \\
            \semantics[ds]{S}{\texttt{if}\,b\,\texttt{then}\,S_1\,\texttt{else}\,S_2} \varphi = & cond(\semantics{B}{b}, \semantics[ds]{S}{S_1}, \semantics[ds]{S}{S_2}) \\
            \semantics[ds]{S}{\texttt{while}\,b\,\texttt{do}\,S} \varphi = & \text{FIX}(\lambda g . cond(\semantics{B}{b}, g \circ \semantics[ds]{S}{S}, id))
        \end{flalign*}
        Where
        \[ cond(pred, g_1, g_2) = \begin{cases}
            g_1(\varphi') & pred(\varphi) = (\mathbf{tt}, \varphi') \\
            g_2(\varphi') & pred(\varphi) = (\mathbf{ff}, \varphi') \\
            \uparrow & \text{otherwise}
        \end{cases} \]
    \end{exampleblock}
\end{frame}