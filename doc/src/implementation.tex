\section{Implementation}

\subsection{The Bounded Intervals Domain}

\begin{frame}{Bounded Intervals - Definition}
    \[ I_{m,n} \subset \wp(\mathbb{Z})\text{ with }m,n\in\mathbb{Z}\cup{-\infty, \infty}\]
    \begin{flalign*}
        I_{m,n} & = \{ \mathbb{Z}, \emptyset \} \cup \{ \{ z \} \pipe z \in \mathbb{Z} \} \\
        & \cup \{ \{ x \pipe w <= x <= z \} \pipe x,w,z \in \mathbb{Z}\text{ s.t. }m <= w <= z <= n \} \\
        & \cup \{ \{ x \pipe x \leq z \} \pipe x,z \in \mathbb{Z}\text{ s.t. }m <= z <= n\} \\
        & \cup \{ \{ x \pipe x \geq z \} \pipe x,z \in \mathbb{Z}\text{ s.t. }m <= z <= n\} \\
    \end{flalign*}
\end{frame}

\begin{frame}{Bounded Intervals - Properties (1)}

    \begin{exampleblock}{$I_{m,n}$ is partially ordered}
        \[ i_1 \leq i_2 \iff i_1 \subseteq i_2 \]
    \end{exampleblock}

    \begin{exampleblock}{$I_{m,n}$ is a complete lattice}
        \begin{flalign*}
            \bot_{I_{m,n}} &= \emptyset \\
            \top_{I_{m,n}} &= \mathbb{Z} \\
            \lor_{I_{m,n}} &= \cup \\
            \land_{I_{m,n}} &= \cap
        \end{flalign*}
    \end{exampleblock}
\end{frame}

\begin{frame}{Bounded Intervals - Properties (2)}
    \begin{itemize}
        \item $I_{m,n}$ has no infinite ascending chains when $m \neq -\infty \land n \neq \infty$:
        \begin{itemize}
            \item when $m,n\in \mathbb{N}$ the fixed-point iteration sequence induced by $\mathcal{D}^{\#}\llbracket S_1 \rrbracket s$ $\forall s \in \mathbb{S}_A,\,\,S_1 \in \textbf{While}$ converges in finite time;
            \item otherwise, we must make use of the widening operator $\nabla: \mathbb{S}_A \to \mathbb{S}_A \to\mathbb{S}_A$ in order to enforce convergence. 
        \end{itemize}
        \item $I_{m,n}$ has no infinite descending chains:
        \begin{itemize}
            \item any descending greatest fixed-point search converges in finite time;
            \item there is no need for a narrowing operator $\Delta: \mathbb{S}_A \to \mathbb{S}_A \to\mathbb{S}_A$.
        \end{itemize}
    \end{itemize}
    
\end{frame}

\subsection{Abstract States}

\begin{frame}{Abstract States (1)}
    We define for any abstract domain $A$, which is a complete lattice as well, the abstract state type $\mathbb{S}_{A} = Map(Var, A)$.
    
    \begin{alertblock}{Assumption}
        When a variable is used before its definition, then its value is assumed to be ``unknown'' ($\top_{A}$). This is due to the fact that we assume that all referenced variables in the program are initialized.
        %% Non è affare nostro sapere se ci sono delle variabili non inizializzate, bensì ci occorre sapere se il programma termini in maniera corretta
    \end{alertblock}
\end{frame}

\begin{frame}{Abstract States (2)}

    Moreover, $\bot_{\mathbb{S}_{A}}$ represents an abnormal termination (no update operation can be performed over this state):

    \begin{flalign*}
        s(x) &= \begin{cases}
            a & (x, a) \in s \\
            \top_{A} & otherwise 
        \end{cases} \\
        s[x \mapsto a] &= \begin{cases}
            \bot_{\mathbb{S}_{A}} & s = \bot_{\mathbb{S}_{A}} \\
            \{ (k, v) \pipe (k, v) \in s,\,\,k\neq x \} & a \neq \top_{A},\,\,s \neq \bot_{\mathbb{S}_{A}}\\
            \{ (k, v) \pipe (k, v) \in s,\,\,k\neq x \}  & otherwise
        \end{cases}
    \end{flalign*}

    
\end{frame}


\begin{frame}{Abstract States (3)}
    \begin{exampleblock}{$\mathbb{S}_{A}$ is partially ordered}
        \[ s_1 \leq_{\mathbb{S}_A} s_2 \iff s_1(x) \leq_A s_2(x)\,\,\forall x \in Var \]
    \end{exampleblock}

    \begin{exampleblock}{$\mathbb{S}_{A}$ is a complete lattice}
        \begin{flalign*}
            \bot_{\mathbb{S}_{A}} &= \{ (x, \bot_{A}) \pipe x \in Var \} \\
            \top_{\mathbb{S}_{A}} &= \emptyset \\
            s_1 \lor_{\mathbb{S}_{A}} s_2  &= \{ (var, a_1 \lor_{A} a_2) \pipe (var, a_1) \in s_1, (var, a_2) \in s_2 \} \\
            s_1 \land_{\mathbb{S}_{A}} s_2  &= \{ (var, a_1 \land_{A} a_2) \pipe (var, a_1) \in s_1, (var, a_2) \in s_2 \} \\
            &\cup \{ e \pipe e \in s_1, e \notin s_2 \} \cup \{ e \pipe e \notin s_1,e \in s_2 \} \\
        \end{flalign*} 
    \end{exampleblock} 
\end{frame}

\subsection{Abstract Semantics}

\begin{frame}{Abstract Semantics}
    The abstract semantic functions are:
    \begin{itemize}
        \item $\mathcal{A}^\sharp : AExp \to \mathbb{S}_{A} \to A \times \mathbb{S}_{A}$:
        \begin{itemize}
            \item the first element of the tuple approximates the possible results of the arithmetic expression;
            \item the second element approximates the possible states after the transition induced by the expression;
        \end{itemize}
        \item $\mathcal{B}^\sharp : BExp \to \mathbb{S}_{A} \to \mathbb{S}_A \times \mathbb{S}_A$;
        \begin{itemize}
            \item the first element of the tuple approximates the states where the boolean expression can evaluate to \textbf{tt};
            \item the second element approximates the states where the boolean expression can evaluate \textbf{ff}.
        \end{itemize}
        This function returns two states, instead of one, in order to preserve the short circuit behavior of boolean operators along with a compositional definition.
        %% per mantenere rappresentato lo short-circuit degli operatori logici mantenendo la definizione composizionale era necessario che la funzione B# restituisse anche lo stato in caso BExp valutasse a falso
        \item $\mathcal{D}^\sharp : While \to \mathbb{S}_{A} \to\mathbb{S}_{A}$.
    \end{itemize}
\end{frame}

\begin{frame}{Abstract Semantics - AExpr (1)}
    \begin{exampleblock}{$\mathcal{A}^\sharp : AExpr \to \mathbb{S}_A \to I_{m,n} \times \mathbb{S}_A$}
        \begin{flalign*}
            \asem{A}{n} s^{\sharp} = & (\{n_\mathbb{Z}\}, s^{\sharp}) \\
            \asem{A}{x} s^{\sharp} = & (s^{\sharp}(x), s^{\sharp}) \\
            \asem{A}{(e)} s^{\sharp} = & \asem{A}{e} s^{\sharp} \\
            \asem{A}{-e} s^{\sharp} = & (-a^{\sharp}, s^{\sharp}_1) \\
            \text{where } & (a^{\sharp}, s^{\sharp}_1) = \asem{A}{e} s^{\sharp} \\
            \asem{A}{[e_1, e_2]} s^{\sharp} = & (\{a^{\sharp}_1,\dots,a^{\sharp}_2\}, s^{\sharp}_2) \\
            \text{where } & (a^{\sharp}_1, s^{\sharp}_1) = \asem{A}{e_1} s^{\sharp} \\
            & (a^{\sharp}_2, s^{\sharp}_2) = \asem{A}{e_2} s^{\sharp}_1
        \end{flalign*}
    \end{exampleblock}
\end{frame}


\begin{frame}{Abstract Semantics - AExpr (2)}
    \begin{exampleblock}{$\mathcal{A}^\sharp : AExpr \to \mathbb{S}_A \to I_{m,n} \times \mathbb{S}_A$}
        \begin{flalign*}
            \asem{A}{e_1\,\mathbf{op}\,e_2} s^{\sharp} = & (a^{\sharp}_1\,\,op_{I_{m,n}}\,\,a^{\sharp}_2, s^{\sharp}_2) \\ 
            \text{where } & (a^{\sharp}_1, s^{\sharp}_1) = \asem{A}{e_1} s^{\sharp} \\
            & (a^{\sharp}_2, s^{\sharp}_2) = \asem{A}{e_2} s^{\sharp}_1
        \end{flalign*}
    \end{exampleblock}
\end{frame}


\begin{frame}{Abstract Semantics - AExpr (3)}
    \begin{exampleblock}{$\mathcal{A}^\sharp : AExpr \to \mathbb{S}_A \to I_{m,n} \times \mathbb{S}_A$}
        \begin{flalign*}
            \asem{A}{x\texttt{++}} s^{\sharp} = & (s^{\sharp}(x), s^{\sharp}[x \mapsto x+_{I_{m,n}}1]) \\
            \asem{A}{\texttt{++}x} s^{\sharp} = & (s^{\sharp}_1(x), s^{\sharp}_1) \\
            \text{where } & s^{\sharp}[x \mapsto x+_{I_{m,n}}1] = s^{\sharp}_1 \\
            \asem{A}{x\texttt{--}} s^{\sharp} = & (s^{\sharp}(x), s^{\sharp}[x \mapsto x-_{I_{m,n}}1]) \\
            \asem{A}{\texttt{--}x} s^{\sharp} = & (s^{\sharp}_1(x), s^{\sharp}_1) \\
            \text{where } & s^{\sharp}[x \mapsto x-_{I_{m,n}}1] = s^{\sharp}_1 \\
        \end{flalign*}
    \end{exampleblock}
\end{frame}

\begin{frame}{Abstract Semantics - BExpr (1)}
    %The first member of the pair over-approximates the states where the boolean expression holds, the second member over-approximates the states where the expression doesn't hold.

    \begin{exampleblock}{$\mathcal{B}^{\sharp} : BExp \to \mathbb{S}_A \to \mathbb{S}_A \times \mathbb{S}_A$}
        \begin{flalign*}
            \asem{B}{\texttt{true}} s^{\sharp} = & (s^{\sharp}, \bot_{\mathcal{S}_A}) \\
            \asem{B}{\texttt{false}} s^{\sharp} = & (\bot_{\mathcal{S}_A}, s^{\sharp}) \\
            \asem{B}{(b)} s^{\sharp} = & \asem{B}{b} s^{\sharp}\\
            \asem{B}{b_1\,\,\texttt{and}\,\,b_2} s^{\sharp} = & \begin{cases}
                (\mathbf{ff}, s^{\sharp}_1) & \asem{B}{b_1} s^{\sharp} = (\mathbf{ff}, s^{\sharp}_1) \\
                \asem{B}{b_2} s^{\sharp}_1 & \asem{B}{b_1} s^{\sharp} = (\mathbf{tt}, s^{\sharp}_1) \\
                \uparrow & \text{otherwise}
            \end{cases}\\
            \asem{B}{b_1\,\,\texttt{or}\,\,b_2} s^{\sharp} = & \begin{cases}
                (\mathbf{tt}, s^{\sharp}_1) & \asem{B}{b_1} s^{\sharp} = (\mathbf{tt}, s^{\sharp}_1) \\
                \asem{B}{b_2} s^{\sharp}_1 & \asem{B}{b_1} s^{\sharp} = (\mathbf{ff}, s^{\sharp}_1) \\
                \uparrow & \text{otherwise}
            \end{cases}\\
        \end{flalign*}
        
    \end{exampleblock}
\end{frame}

\begin{frame}{Abstract Semantics - BExpr (2)}
    Comparison operators propagate the state transition(s):

    \begin{exampleblock}{$\mathcal{B} : BExp \to State \hookrightarrow \mathbb{T} \times State$}
        \begin{flalign*}
            \semantics{B}{e_1\texttt{ = }e_2} \varphi = &\begin{cases}
                (a_1 = a_2, \varphi'') & \semantics{A}{e_1} \varphi = (a_1, \varphi') \\
                & \land \,\, \semantics{A}{e_2} \varphi' = (a_2, \varphi'') \\
                \uparrow & \text{otherwise}
            \end{cases}\\
            \semantics{B}{e_1\texttt{ != }e_2} \varphi = &\begin{cases}
                (a_1 \neq a_2, \varphi'') & \semantics{A}{e_1} \varphi = (a_1, \varphi') \\
                & \land \,\, \semantics{A}{e_2} \varphi' = (a_2, \varphi'') \\
                \uparrow & \text{otherwise}
            \end{cases}\\
        \end{flalign*}
    \end{exampleblock}
\end{frame}


\begin{frame}{Abstract Semantics - BExpr (3)}
    \begin{exampleblock}{$\mathcal{B} : BExp \to State \hookrightarrow \mathbb{T} \times State$}
        \begin{flalign*}
            \semantics{B}{e_1\texttt{ < }e_2} \varphi = & \begin{cases}
                (a_1 < a_2, \varphi'') & \semantics{A}{e_1} \varphi = (a_1, \varphi') \\
                & \land \,\, \semantics{A}{e_2} \varphi' = (a_2, \varphi'') \\
                \uparrow & \text{otherwise}
            \end{cases}\\
            \semantics{B}{e_1\texttt{ >= }e_2} \varphi = & \begin{cases}
                (a_1 \geq a_2, \varphi'') & \semantics{A}{e_1} \varphi = (a_1, \varphi') \\
                & \land \,\, \semantics{A}{e_2} \varphi' = (a_2, \varphi'') \\
                \uparrow & \text{otherwise}
            \end{cases}\\
        \end{flalign*}
    \end{exampleblock}
\end{frame}

\begin{frame}{Abstract Semantics - BExpr (4)}
    \begin{exampleblock}{$\mathcal{B} : BExp \to State \hookrightarrow \mathbb{T} \times State$}
        \begin{flalign*}
            \semantics{B}{e_1\texttt{ > }e_2} \varphi = &\begin{cases}
                (a_1 > a_2, \varphi'') & \semantics{A}{e_1} \varphi = (a_1, \varphi') \\
                & \land \,\, \semantics{A}{e_2} \varphi' = (a_2, \varphi'') \\
                \uparrow & \text{otherwise}
            \end{cases}\\\\
            \semantics{B}{e_1\texttt{ <= }e_2} \varphi = & \begin{cases}
                (a_1 \leq a_2, \varphi'') & \semantics{A}{e_1} \varphi = (a_1, \varphi') \\
                & \land \,\, \semantics{A}{e_2} \varphi' = (a_2, \varphi'') \\
                \uparrow & \text{otherwise}
            \end{cases}\\
        \end{flalign*}
    \end{exampleblock}
\end{frame}

\begin{frame}{Abstract Semantics - Statements (1)}
    \begin{exampleblock}{$\mathcal{D}^\sharp : \mathbf{While} \to \mathbb{S}_{A} \to \mathbb{S}_{A}$}
        
        \begin{flalign*}
            \asem{D}{x\texttt{ := }e} s^{\sharp} & \defas
            \begin{cases}
                s'^\sharp [x \mapsto a] & (a, s'^\sharp) = \asem{A}{e} s^\sharp\\ 
                & \land a \neq \bot_A \\
                \bot_{\mathbb{S}_{A}} & otherwise
            \end{cases}\\
            \asem{D}{\texttt{skip}} s^{\sharp} & \defas s^{\sharp} \\
            \asem{D}{S_1\texttt{ ; }S_2} s^{\sharp} & \defas (\asem{D}{S_1} \circ \asem{D}{S_2}) s^{\sharp}
        \end{flalign*}
    \end{exampleblock}
\end{frame}

\begin{frame}{Abstract Semantics - Statements (2)}
    \begin{exampleblock}{$\mathcal{D}^\sharp : \mathbf{While} \to \mathbb{S}_{A} \to \mathbb{S}_{A}$}
        
        \small\begin{flalign*}
            \asem{D}{\texttt{if }b\texttt{ then }S_1\texttt{ else }S_2} s^{\sharp} & \defas (\asem{B}{S_1} s^{\sharp}_{\mathbf{tt}}) \lor_{\mathbb{S}_A} (\asem{B}{S_2} s_{\mathbf{ff}}^{\sharp}) \\
            where &\quad\,\,\, (s^{\sharp}_{\mathbf{tt}}, s_{\mathbf{ff}}^{\sharp}) = \asem{B}{b} s^{\sharp}\\
            \asem{D}{\texttt{while }b\texttt{ do }S} s^{\sharp} & \defas \pi_2 (\asem{B}{b} (\text{GFP}_{\text{FIX } F} (\lambda s.s \land_{\mathbb{S}_A} F\,\,s))) \\
            where &\quad\,\,\, F: \mathbb{S}_A \to \mathbb{S}_A \\
            &\quad\,\,\, F\,\,s = s^{\sharp} \lor_{\mathbb{S}_A} (\asem{D}{S} \circ \pi_1 \circ \asem{B}{b} s) \\
        \end{flalign*}
    \end{exampleblock}

    \small{Where FIX $F$ refers to the fixed point of the function $F$ and GFP$_s$ $f$ is the greatest fixed point of $f$ found starting from $s$.}
\end{frame}

\begin{frame}{Abstract Semantics - Statements (3)}
    \small\begin{alertblock}{Widened invariant refinition}
        Since $I_{m,n}$ has infinitely ascending chains, FIX $F$ might diverge. Therefore, in the implementation, we make use of a widened iteration sequence.

        The \textbf{widened invariant} resulting from (possibly) widened FIX $F$ is later refined with the GFP of $\lambda s.s \land_{\mathbb{S}_A} F\,s$.
    \end{alertblock}

    \small{This is sound:
    \begin{itemize}
        \item the widened invariant $s^{*\sharp}$ is a sound over-approximation of the smallest loop invariant $s^{*}$;
        \item $s^{*} = F\,\,s^{*}$, so $s^{*} = s^{*} \land_{\mathbb{S}_A} F\,\,s^{*}$: therefore $\lambda s.s \land_{\mathbb{S}_A} F\,s$ (descending monotone) is a sound filtering of those states not in $s^*$. 
    \end{itemize}
    Therefore, GFP $F$ starting from $s^{*\sharp}$ is the most precise refinement of the widened invariant.}
    
\end{frame}